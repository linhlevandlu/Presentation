\documentclass{beamer}
\usepackage{pgfpages}
\usepackage[utf8]{inputenc}
\usepackage{times}
\usepackage{tikz}
\usetheme{Warsaw}
\setbeamercovered{transparent}
\useoutertheme{infolines}
\setbeamertemplate{footline} [frame number]
\title{Automatic identification of landmarks by shape recognize}
\author{LE Van Linh}

\date{Oct 29, 2015}
\begin{document}
\frame{\titlepage}
\begin{frame}{Contents}
	\tableofcontents
\end{frame}
\section{Introduction}
\begin{frame}{Introduction}
Introduction about presentation
\end{frame}
\begin{frame}{Introduction}
The flow diagram about the steps
\end{frame}
\section{Method}
\subsection{Segmentation}
\begin{frame}{Segmentation}
	Purpose: 
	\begin{itemize}
		\item Extract the features (edge) from images
		\item Get the approximate lines
	\end{itemize}
	Method:
	\begin{itemize}
		\item Indicate the threshold value by analysis histogram of image		
		\item Canny
		\item Break edge algorithm
	\end{itemize}
	Result: The set of approximate lines
\end{frame}
\subsection{Pairwise geometric histogram(PGH)}
\begin{frame}{Pairwise geometric histogram}
	\begin{description}
		\item[Purpose:] detecting the present of scene image in model image
		\item[Method:]
		\begin{itemize}
			\item Construct the local PGH
			\item Construct the shape PGH
			\item Matching shape's PGH by Bhattacharyya metric
		\end{itemize}
	\end{description}
\end{frame}
\begin{frame}{Pairwise geometric histogram}
	\framesubtitle{Local PGH and shape PGH}
	\begin{description}
		\item[PGH:] a matrix two dimensions: angle axis and distance axis
		\item[PGH information:] angle between two lines and perpendicular distance from two endpoints of scene line to reference line.
		\item[Local PGH]: PGH for each feature (line)
		\item[Shape PGH]: contains many \textbf{Local PGH}
	\end{description}
\end{frame}
\subsection{Probabilistic Hough Transform}
\begin{frame}{Probabilistic Hough Transform}
	\begin{description}
		\item[Purpose:]
			\begin{itemize}
				\item Determine the presence and location of model image in scene image
				\item Estimate the landmarks in the scene image
			\end{itemize}
		\item[Method:]
		\begin{itemize}
			\item Construct the reference table
			\item Find the pair scene lines have the best ``vote"
			\item Estimate the ``reference point" in scene image
			\item Estimate the landmarks
		\end{itemize}
	\end{description}
\end{frame}
\subsection{Template matching}
\begin{frame}{Template matching}
	This slide talk about the PHT method
\end{frame}
\section{Result}
\begin{frame}{Result}
	This slide talk about the result
\end{frame}
\section{Conclusion}
\begin{frame}{Conclusion}
	This slide talk about conclusion
\end{frame}
\section{References}
\begin{frame}{References}
	References slide
\end{frame}
\end{document}